\documentclass{beamer}
\usepackage{etex}
\usetheme{default}
%\useinnertheme{rounded}
\usecolortheme{seagull}
\usefonttheme{structurebold}
%\usefonttheme{professionalfonts}
\usenavigationsymbolstemplate{}
%\setbeamerfont{frametitle}{family=\bfseries}

\definecolor{paprika}{rgb}{0.6,0,0.2}
\definecolor{pineglade}{rgb}{0.95,0.95,0.75}

\setbeamercolor{structure}{fg=paprika}
\setbeamercolor{block title}{fg=paprika,bg=pineglade}
\setbeamercolor{alerted text}{fg=paprika}
\setbeamercolor{block body}{bg=pineglade}
\setbeamertemplate{title page}
{%
   \begin{center}
      \begin{minipage}{0.9\textwidth}
         \begin{block}{}
            \begin{center}
              \Large{\alert{\inserttitle}}\\
              \large{\alert{\insertsubtitle}}\\
               \vspace{0.5cm}
               \large{\insertauthor}\\
               \large{\texttt{melissa.mendonca@ufsc.br}}
            \end{center}
         \end{block}
      \end{minipage}
   \end{center}
}

\usepackage[portuguese]{babel}
%\usepackage{fourier}
\usepackage{tikz}
\usetikzlibrary{positioning}
\usetikzlibrary{arrows, decorations.markings}
% for double arrows a la chef
\tikzstyle{vecArrow} = [thick, paprika, decoration={markings,mark=at position
   1 with {\arrow[semithick,paprika]{open triangle 60}}},
   double distance=1.4pt, shorten >= 5.5pt,
   preaction = {decorate},
   postaction = {draw,line width=1.4pt, pineglade,shorten >= 4.5pt}]
% Define box and box title style
\tikzstyle{mybox} = [draw, very thick, rectangle,rounded corners, inner sep=10pt, inner ysep=20pt]
\usepackage{ctable} % provides \toprule,\bottomrule,\midrule for tables

\newcommand{\kw}[1]{\alert{\texttt{#1}}}
\newcommand{\code}[1]{{\texttt{#1}}}
\newcommand{\acode}[1]{\alert{\texttt{#1}}}
\newcommand{\codigo}[1]{\begin{center}\rm{\code{
  \begin{tabular}{r l}
  #1
  \end{tabular}
  }}\end{center}}
\newcommand{\ac}{\alert{\texttt{>>}}}
\newcommand{\file}[3]{\texttt{
   \begin{center}
      \begin{tikzpicture}
         \node[mybox] (box) {
            \begin{minipage}{#1}
               #3
            \end{minipage}
         };
         \node[draw, fill=white, text=black, right=10pt, rounded corners] at (box.north west) {
            \textbf{#2}
         };
      \end{tikzpicture}
   \end{center}}
}

\title{MATLAB Avançado}
\subtitle{Aula 2}
\author[M. Weber Mendonça]{Melissa Weber Mendonça}
\institute[UFSC]{\inst{1} Universidade Federal de Santa Catarina} 
\date{2014.1}

\logo{\includegraphics[height=1.5cm]{img/brasao_UFSC.png}}

\begin{document}
%%%
% Slide 1
\begin{frame}
% Title
  \titlepage
\end{frame}

\begin{frame}
  \frametitle{Estruturas}
  Outra maneira de armazenar dados heterogêneos é usar \emph{estruturas}: cada estrutura é composta de campos que podem conter quaisquer tipos de dados (assim como as células), e que são referenciados por \emph{nome}. Para criarmos uma estrutura chamada \code{dados} com o campo chamado \code{Nome}, podemos usar diretamente a sintaxe\\
  \vfill
  \codigo{\ac & dados.Nome = 'Melissa'\\
    \ac & dados.Sobrenome = 'Mendonca'}
  ou
  \codigo{\ac & dados = \acode{struct}('Nome', 'Melissa', ...\\
    & \qquad 'Sobrenome', 'Mendonca')}
\end{frame}

\begin{frame}
   \frametitle{Exemplos de uso}
   \codigo{\ac & dados = \acode{struct}('Nome', 'Melissa', ...\\
     & \qquad 'Sobrenome', 'Mendonça')\\
     \ac & dados(1)\\
     \ac & dados(1).Nome\\
     \ac & dados(2) = \acode{struct}('Nome', 'Fulano', ...\\
     & \qquad 'Sobrenome', 'Beltrano')\\
     \ac & dados(1)\\
     \ac & dados(2)\\
     \ac & dados.Nome\\
     \ac & [nome1, nome2] = dados.Nome}

   \textbf{Obs.} Para criarmos uma struct vazia, podemos usar o comando
   \codigo{\ac & vazia = \acode{struct}([])}
\end{frame}
%%%%%%%%%%%%%%%%%%%%%%%%%%%%%%%%%%%%%%%%%%%%%%%%%%%%%%%%%%%%%%%%%%%%%%%%%%%%%%%%%%%%%%%%%
\begin{frame}
   \frametitle{Campos}
   As structs possuem campos nomeados, o que pode tornar mais fácil acessar os dados armazenados nesse tipo de variável. Alguns comandos do MATLAB permitem fazer isso.

   \begin{itemize}
      \item<1-> O comando \code{\acode{fieldnames}(s)} permite recuperar em uma célula a lista dos nomes dos campos da struct \code{s}.
      \item<2-> O comando \code{s = \acode{orderfields}(s1)} ordena os campos da struct \code{s1} de modo que a nova struct \code{s} tem os campos em ordem alfabética.
      \item<3-> O comando \code{s = \acode{orderfields}(s1, s2)} ordena os campos da struct \code{s1} de forma que a nova struct \code{s} tenha os nomes dos campos na mesma ordem em que aparecem na struct \code{s2} (as structs \code{s1} e \code{s2} devem ter os mesmos campos).
      \item<4-> O comando \code{s = \acode{orderfields}(s1, c)} ordena os campos em \code{s1} de forma que a nova struct \code{s} tenha campos na mesma ordem em que aparecem na célula \code{c} (a célula \code{c} deve conter apenas os nomes dos campos de \code{s1}, em qualquer ordem).
   \end{itemize}
\end{frame}
%%%%%%%%%%%%%%%%%%%%%%%%%%%%%%%%%%%%%%%%%%%%%%%%%%%%%%%%%%%%%%%%%%%%%%%%%%%%%%%%%%%%%%%%%
\begin{frame}
   \frametitle{Structs e células}

   Podemos preencher uma struct usando um comando só, determinando os valores possíveis para cada campo através de células. Por exemplo, no caso anterior, poderíamos ter entrado o comando
   \codigo{\ac & dados = \acode{struct}('Nome', \{'Melissa', 'Fulano'\}, ...\\
     & \qquad 'Sobrenome', \{'Mendonça', 'Beltrano'\})}
   para criar a mesma struct.

   Se quisermos preencher vários campos com o mesmo valor, não precisamos nos repetir. Por exemplo, 
   \codigo{\ac & dados = \acode{struct}('Nome', \{'Melissa', 'Fulano'\}, ...\\
     & \qquad 'Sobrenome', \{'Mendonça', 'Beltrano'\}, ...\\
     & \qquad 'Presentes', \{'sim'\})}
\end{frame}
%%%%%%%%%%%%%%%%%%%%%%%%%%%%%%%%%%%%%%%%%%%%%%%%%%%%%%%%%%%%%%%%%%%%%%%%%%%%%%%%%%%%%%%%%
\begin{frame}
  \frametitle{\acode{cell2struct}}
  A função \acode{cell2struct} cria uma estrutura a partir dos dados contidos na célula:

  Se Exemplo:
  \codigo{\ac & tabela = \{'Melissa', 'Mendonça', 'sim';\\
    & \qquad 'Fulano', 'Beltrano', 'sim'\}\\
    \ac & campos = \{'Nome', 'Sobrenome', 'Presente'\};\\
    \ac & s = \acode{cell2struct}(tabela, campos, 2);\\
    \ac & s(1)\\
    \ac & s(2)}
\end{frame}
%%%%%%%%%%%%%%%%%%%%%%%%%%%%%%%%%%%%%%%%%%%%%%%%%%%%%%%%%%%%%%%%%%%%%%%%%%%%%%%%%%%%%%%%%
\begin{frame}
   \frametitle{\acode{struct2cell}}
   Por outro lado, o comando
   \codigo{\ac & célula = \acode{struct2cell}(\rm\emph{struct})}
   cria uma célula a partir da estrutura \emph{struct}.
   
   Exemplo:
   \codigo{\ac & celula = \acode{struct2cell}(s)}
\end{frame}
%%%%%%%%%%%%%%%%%%%%%%%%%%%%%%%%%%%%%%%%%%%%%%%%%%%%%%%%%%%%%%%%%%%%%%%%%%%%%%%%%%%%%%%%%
\begin{frame}
  \frametitle{}
  \begin{center}
    \alert{Funções}
  \end{center}
\end{frame}
%%%%%%%%%%%%%%%%%%%%%%%%%%%%%%%%%%%%%%%%%%%%%%%%%%%%%%%%%%%%%%%%%%%%%%%%%%%%%%%%%%%%%%%%%
\begin{frame}
   \frametitle{Funções}
   Na matemática, 
	$$f(x) = y.$$
	
	Entrada: $x$
	
	Saída: $y$
	
	Ação: $f$.
	
	Exemplo: $f(x) = x^2$.
\end{frame}
%%%%%%%%%%%%%%%%%%%%%%%%%%%%%%%%%%%%%%%%%%%%%%%%%%%%%%%%%%%%%%%%%%%%%%%%%%%%%%%%%%%%%%%%%
\begin{frame}
   \frametitle{Funções já prontas}
   Exemplos:

   \codigo{& n = \acode{input}('Entre com um numero:')\\
      & nfat = \acode{factorial}(n)\\
      & texto = \acode{num2str}(25)}
\end{frame}
%%%%%%%%%%%%%%%%%%%%%%%%%%%%%%%%%%%%%%%%%%%%%%%%%%%%%%%%%%%%%%%%%%%%%%%%%%%%%%%%%%%%%%%%%
\begin{frame}
  \frametitle{Funções}

  No MATLAB, uma função é um arquivo \code{minhafuncao.m} com a sintaxe

  \file{0.6\textwidth}{minhafuncao.m}{
    \codigo{function &[y] = minhafuncao(x)\\
      & \% Descricao da funcao\\
      & comandos;
    }}
  
  Uma vez construida a função, podemos chamá-la no console, usando
  \codigo{\ac & y = minhafuncao(x)}

  {\bf{Observação.}} Uma função deve sempre ter o mesmo nome que\\ o arquivo no qual ela está salva.
\end{frame}
%%%%%%%%%%%%%%%%%%%%%%%%%%%%%%%%%%%%%%%%%%%%%%%%%%%%%%%%%%%%%%%%%%%%%%%%%%%%%%%%%%%%%%%%%
\begin{frame}[fragile]
  \frametitle{Exemplo}
  Construir uma função que calcule a média dos 3 elementos de um vetor $x$.

  \file{0.8\textwidth}{media.m}{
    \codigo{function &[y] = media(x)\\
      & y = (x(1) + x(2) + x(3))/3;}}
\end{frame}
%%%%%%%%%%%%%%%%%%%%%%%%%%%%%%%%%%%%%%%%%%%%%%%%%%%%%%%%%%%%%%%%%%%%%%%%%%%%%%%%%%%%%%%%%
\begin{frame}
   \frametitle{Qual a diferença entre um \emph{script} e uma \emph{função}?}
   Um \emph{script} é um arquivo que contém uma sequência de comandos, mas não exige entrada ou saída.
   
   Uma função deve, obrigatoriamente, ter pelo menos uma entrada e uma saída.
\end{frame}
%%%%%%%%%%%%%%%%%%%%%%%%%%%%%%%%%%%%%%%%%%%%%%%%%%%%%%%%%%%%%%%%%%%%%%%%%%%%%%%%%%%%%%%%%
\begin{frame}
   \frametitle{Argumentos de entrada e saída}
   Se tivermos mais de um argumento de entrada, basta separá-los por vírgulas:
   
   \codigo{&s = \acode{soma}(x,y)}
   
   Se tivermos mais de um argumento de saída, precisamos escrevê-los entre colchetes:
   
   \codigo{[a,b] = \acode{somaesubtracao}(x,y)}

   Se quisermos aplicar a mesma função a um conjunto de valores, basta colocarmos os valores em um vetor:
   
   \codigo{&m = \acode{f}([-2 1 3])}
\end{frame}
%%%%%%%%%%%%%%%%%%%%%%%%%%%%%%%%%%%%%%%%%%%%%%%%%%%%%%%%%%%%%%%%%%%%%%%%%%%%%%%%%%%%%%%%%
\begin{frame}
   \frametitle{Funções com número variável de argumentos}
   \begin{itemize}
      \item O comando \acode{nargin}, executado dentro do corpo de uma função, retorna o número de argumentos de entrada para o qual a função está definida.
      \item O comando \acode{nargin(f)}, em que \code{f} é uma função, retorna o número de argumentos de entrada da função \code{f}, e pode ser executado fora da função (inclusive no console).
      \item O comando \acode{nargout}, executado dentro do corpo de uma função, retorna o número de argumentos de saída para o qual a função está definida.
      \item O comando \acode{nargin(f)}, em que \code{f} é uma função, retorna o número de argumentos de saída da função \code{f}, e pode ser executado fora da função (inclusive no console).
   \end{itemize}
\end{frame}
%%%%%%%%%%%%%%%%%%%%%%%%%%%%%%%%%%%%%%%%%%%%%%%%%%%%%%%%%%%%%%%%%%%%%%%%%%%%%%%%%%%%%%%%%
\begin{frame}
   \frametitle{Funções com número variável de argumentos}
   \acode{varargin} é uma variável de entrada que permite que a função receba qualquer número de argumentos de entrada.
   
   Exemplo:
   \file{0.8\textwidth}{somas.m}{\begin{tabular}{ll}function [y] = somas(x,varargin)\\
        \qquad if nargin == 1\\
        \qquad \quad disp('Nada a calcular.')\\
        \qquad elseif nargin == 2\\
        \qquad \quad y = x + varargin\{1\};\\
        \qquad elseif nargin == 3\\
        \qquad \quad y = x + varargin\{1\} + varargin\{2\};\\
        \qquad else\\
        \qquad \quad disp('Argumentos demais!')\\
        \qquad end\end{tabular}}
\end{frame}
%%%%%%%%%%%%%%%%%%%%%%%%%%%%%%%%%%%%%%%%%%%%%%%%%%%%%%%%%%%%%%%%%%%%%%%%%%%%%%%%%%%%%%%%%
\begin{frame}
   \frametitle{Funções com número variável de argumentos}
   \acode{varargout} é uma variável de saída que permite que a função devolva qualquer número de argumentos de saída.
   
   Exemplo:
   \file{0.8\textwidth}{valores.m}{\begin{tabular}{l} function [f,varargout] = valores(x)\\
        \qquad f = x\^{}2;\\
        \qquad  if nargout == 2\\
        \qquad  \quad varargout\{1\} = 2*x;\\
        \qquad elseif nargout == 3\\
        \qquad \quad varargout\{1\} = 2*x;\\
        \qquad \quad varargout\{2\} = 2;\\
        \qquad elseif nargout $>$ 3\\
        \qquad \quad disp(['A função aceita até 3 ' ... \\
        \qquad \quad \quad 'argumentos de saída.'])\\
        \qquad end\end{tabular}}
\end{frame}
%%%%%%%%%%%%%%%%%%%%%%%%%%%%%%%%%%%%%%%%%%%%%%%%%%%%%%%%%%%%%%%%%%%%%%%%%%%%%%%%%%%%%%%%%
\begin{frame}
  \frametitle{Funções anônimas}

  Para declarar uma função no console, sem ter que guardá-la em um arquivo, podemos usar o conceito de \emph{função anônima}. 

  Exemplo:
  \codigo{\ac & f = @(x) \acode{sin}(x)\\
    \ac & x = pi;\\
    \ac & f(x)}
  
  Exemplo:
  \codigo{\ac & f = @(x) x-1;\\
    \ac & \acode{min}(f([-2 1 0]))}
\end{frame}
%%%%%%%%%%%%%%%%%%%%%%%%%%%%%%%%%%%%%%%%%%%%%%%%%%%%%%%%%%%%%%%%%%%%%%%%%%%%%%%%%%%%%%%%%
\begin{frame}
  \frametitle{Funções anônimas com mais de uma variável}
  Se quisermos criar uma função anônima com mais de uma variável, usamos 

  \codigo{\ac & f = @(x,y,z,t) x+y+z+t}

  Para retornarmos mais de um valor de uma função anônima, usamos o comando \emph{deal}:

  \codigo{\ac & f = @(t,u,v) \acode{deal}(t+u+v,t-u+2*v)\\
    \ac & [a,b] = f(1,2,3)}
\end{frame}
%%%%%%%%%%%%%%%%%%%%%%%%%%%%%%%%%%%%%%%%%%%%%%%%%%%%%%%%%%%%%%%%%%%%%%%%%%%%%%%%%%%%%%%%%
\begin{frame}
  \frametitle{Funções anônimas: Exemplo}

  Pra que servem funções anônimas?

   \codigo{\ac & x = -3:0.1:3;\\
     \ac & f = @(x) x.\^{}2+3*x\\
     \ac & \acode{plot}(x,f(x));
   }
\end{frame}
%%%%%%%%%%%%%%%%%%%%%%%%%%%%%%%%%%%%%%%%%%%%%%%%%%%%%%%%%%%%%%%%%%%%%%%%%%%%%%%%%%%%%%%%%
\begin{frame}
  \frametitle{}
  \begin{center}
    \alert{Manipulação de Arquivos e Tratamento de Dados}
  \end{center}
\end{frame}
%%%%%%%%%%%%%%%%%%%%%%%%%%%%%%%%%%%%%%%%%%%%%%%%%%%%%%%%%%%%%%%%%%%%%%%%%%%%%%%%%%%%%%%%%
\begin{frame}
   \frametitle{Importar dados}
   Para importarmos dados, o método mais fácil é utilizar a interface gráfica do MATLAB, selecionando 
  
   \begin{center}
      File $\to$ Import Data
   \end{center}

   Para verificar os tipos de arquivo suportados e as funções disponíveis, consulte o Help.
\end{frame}
%%%%%%%%%%%%%%%%%%%%%%%%%%%%%%%%%%%%%%%%%%%%%%%%%%%%%%%%%%%%%%%%%%%%%%%%%%%%%%%%%%%%%%%%%
\begin{frame}
   \frametitle{\acode{importdata}}
   Para importarmos dados de maneira automática, podemos usar o comando 
   \codigo{\ac & \acode{importdata}\code{(arquivo, separador, ncabecalho)}}
\end{frame}
%%%%%%%%%%%%%%%%%%%%%%%%%%%%%%%%%%%%%%%%%%%%%%%%%%%%%%%%%%%%%%%%%%%%%%%%%%%%%%%%%%%%%%%%%
\begin{frame}
  \frametitle{Leitura de dados numéricos: \code{load}}
  
  Para lermos um arquivo com dados \alert{numéricos} chamado \code{dados.txt}, usamos o comando
  
  \codigo{\ac & \acode{load} dados.txt;}

  Em seguida, na variável \code{dados} estarão contidos os valores obtidos do arquivo \code{dados.txt}. 

  Se quisermos também podemos usar a sintaxe

  \codigo{\ac & A = \acode{load}('dados.txt')}

\end{frame}
%%%%%%%%%%%%%%%%%%%%%%%%%%%%%%%%%%%%%%%%%%%%%%%%%%%%%%%%%%%%%%%%%%%%%%%%%%%%%%%%%%%%%%%%%
\begin{frame}
   \frametitle{Abrir e fechar um arquivo}
   Para abrir um arquivo chamado \code{nome.txt}, usamos o comando
   \begin{center}
      \code{arquivo = \acode{fopen}('nome.txt')}
   \end{center}
   \vfill
   Sempre que abrimos um arquivo, precisamos fechá-lo antes de sair do nosso programa. Para isso, usamos o comando
   \begin{center}
      \code{\acode{fclose}(arquivo)}
   \end{center}   
\end{frame}
%%%%%%%%%%%%%%%%%%%%%%%%%%%%%%%%%%%%%%%%%%%%%%%%%%%%%%%%%%%%%%%%%%%%%%%%%%%%%%%%%%%%%%%%%
\begin{frame}
   \frametitle{Comandos: leitura}
   Para ler dados de um arquivo, precisamos indicar que tipo de informação estamos procurando. Isto é feito através dos \emph{formatos} abaixo:
   \begin{itemize}
      \item Números inteiros: \code{\%d} ou \code{\%u}
      \item Números reais: \code{\%f} (notação decimal) ou \code{\%e} (notação científica)
      \item Texto com espaços: \code{\%c}
      \item Texto sem espaços: \code{\%s}
      \item Nova linha: \code{\textbackslash n} (sinaliza o fim de uma linha de dados)
   \end{itemize}

   Para lermos dados de um arquivo em uma célula, usamos
   \begin{center}
      \code{C = \acode{textscan}(arquivo,'\%d')}
   \end{center}
   Para lermos dados de um arquivo em uma matriz, usamos
   \begin{center}
      \code{A = \acode{fscanf}(arquivo,'\%d')}
   \end{center}
\end{frame}
%%%%%%%%%%%%%%%%%%%%%%%%%%%%%%%%%%%%%%%%%%%%%%%%%%%%%%%%%%%%%%%%%%%%%%%%%%%%%%%%%%%%%%%%%
\begin{frame}
   \frametitle{Exemplo}
   \begin{itemize}
      \item[1.] Crie um arquivo chamado
      \begin{center}
         \code{info.txt}
      \end{center}
      no mesmo diretório em que está salvando seus programas, com um número inteiro dentro.
      \item[2.] No console, faça
      \begin{itemize}
         \item[\ac] \code{arquivo = \acode{fopen}('info.txt')}
         \item[\ac] \code{a = \acode{fscanf}(arquivo,'\%d')}
         \item[\ac] \code{\acode{fclose}(arquivo)}
      \end{itemize}
   \end{itemize}

   Verifique que a variável \alert{\code{a}} vale o mesmo que seu inteiro no arquivo.
\end{frame}
%%%%%%%%%%%%%%%%%%%%%%%%%%%%%%%%%%%%%%%%%%%%%%%%%%%%%%%%%%%%%%%%%%%%%%%%%%%%%%%%%%%%%%%%%
\end{document}
