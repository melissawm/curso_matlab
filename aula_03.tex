\documentclass{beamer}
\usepackage{etex}
\usetheme{default}
\usecolortheme{seagull}
\usefonttheme{structurebold}
\usenavigationsymbolstemplate{}

\definecolor{paprika}{rgb}{0.6,0,0.2}
\definecolor{pineglade}{rgb}{0.95,0.95,0.75}

\setbeamercolor{structure}{fg=paprika}
\setbeamercolor{block title}{fg=paprika,bg=pineglade}
\setbeamercolor{alerted text}{fg=paprika}
\setbeamercolor{block body}{bg=pineglade}
\setbeamertemplate{title page}
{%
   \begin{center}
      \begin{minipage}{0.9\textwidth}
         \begin{block}{}
            \begin{center}
               \Large{\alert{\inserttitle}}\\
               \vspace{0.5cm}
               \large{\insertauthor}\\
               \large{\texttt{melissa.mendonca@ufsc.br}}
            \end{center}
         \end{block}
      \end{minipage}
   \end{center}
}

\usepackage[portuguese]{babel}
\usepackage[utf8]{inputenc} % To use characters such as é without typing é
\usepackage{ctable} % provides \toprule,\bottomrule,\midrule for tables
\usepackage{tikz}
\usetikzlibrary{positioning}
\usetikzlibrary{arrows, decorations.markings}
% for double arrows a la chef
\tikzstyle{vecArrow} = [thick, paprika, decoration={markings,mark=at position
   1 with {\arrow[semithick,paprika]{open triangle 60}}},
   double distance=1.4pt, shorten >= 5.5pt,
   preaction = {decorate},
   postaction = {draw,line width=1.4pt, pineglade,shorten >= 4.5pt}]
% Define box and box title style
\tikzstyle{mybox} = [draw, very thick, rectangle,rounded corners, inner sep=10pt, inner ysep=20pt]

\newcommand{\kw}[1]{\alert{\texttt{#1}}}
\newcommand{\code}[1]{{\texttt{#1}}}
\newcommand{\acode}[1]{\alert{\texttt{#1}}}
\newcommand{\codigo}[1]{\begin{center}\rm{\code{
  \begin{tabular}{r l}
  #1
  \end{tabular}
  }}\end{center}}
\newcommand{\ac}{\alert{\texttt{>>}}}
\newcommand{\file}[3]{\texttt{
   \begin{center}
      \begin{tikzpicture}
         \node[mybox] (box) {
            \begin{minipage}{#1}
               #3
            \end{minipage}
         };
         \node[draw, fill=white, text=black, right=10pt, rounded corners] at (box.north west) {
            \textbf{#2}
         };
      \end{tikzpicture}
   \end{center}}
}

\title{MATLAB Avançado} 
\author[M. Weber Mendonça]{Melissa Weber Mendonça}
\institute[UFSC]{\inst{1} Universidade Federal de Santa Catarina} 
\date{2014.1}

\logo{\includegraphics[height=1.5cm]{/home/melissa/Documentos/latex/brasao_UFSC.png}}

\begin{document}
%%%
% Slide 1
\begin{frame}
% Title
  \titlepage
\end{frame}
%%%%%%%%%%%%%%%%%%%%%%%%%%%%%%%%%%%%%%%%%%%%%%%%%%%%%%%%%%%%%%%%%%%%%%%%%%%%%%%%%%%%%%%%%
\begin{frame}
   \frametitle{Relembrando...}
   \begin{itemize}
      \item[\ac] \code{inteiro = \acode{fscanf}(arquivo,’\%d’)}
   \end{itemize}
   Repita o exercício da aula anterior (ler UM dado de um arquivo), agora com um número real:
   \begin{itemize}
      \item[\ac] \code{real = \acode{fscanf}(arquivo,’\%f’)}
   \end{itemize}
   e depois repita com um texto:
   \begin{itemize}
      \item[\ac] \code{texto = \acode{fscanf}(arquivo,’\%s’)}
   \end{itemize}
   \vfill
   Tente colocar o formato errado e observe o erro produzido.
\end{frame}
%%%%%%%%%%%%%%%%%%%%%%%%%%%%%%%%%%%%%%%%%%%%%%%%%%%%%%%%%%%%%%%%%%%%%%%%%%%%%%%%%%%%%%%%%
\begin{frame}
   \frametitle{Lista de dados}
   Se quisermos ler uma lista de números inteiros, por exemplo, basta informar o \emph{padrão} dos dados. 

   Exemplo: se no arquivo temos
   \begin{center}
      \code{1 2 3 4 5}
   \end{center}
   podemos usar o comando
   \begin{center}
      \code{v = \acode{fscanf}(arquivo,'\%d')}
   \end{center}
   \code{v} será um vetor \emph{coluna}. 
\end{frame}
%%%%%%%%%%%%%%%%%%%%%%%%%%%%%%%%%%%%%%%%%%%%%%%%%%%%%%%%%%%%%%%%%%%%%%%%%%%%%%%%%%%%%%%%%
\begin{frame}[fragile]
   \frametitle{Exercício}
   Ler uma matriz 3$\times$3 de dados de um arquivo .txt
   \begin{center}
      \alert{Cuidado com a ordem dos dados!}
   \end{center}

   \file{0.4\textwidth}{matriz.txt}{\code{1 2 3\\4 5 6\\7 8 9}}

   \begin{center}
      \begin{tikzpicture}
         \node[fill=pineglade] (caixa) {
           \begin{minipage}{7cm}
\begin{verbatim}
arquivo = fopen('matriz.txt');
A = fscanf(arquivo,'%f');
fclose(arquivo);
A
\end{verbatim}
           \end{minipage}
        };
      \end{tikzpicture}
   \end{center}
\end{frame}
%%%%%%%%%%%%%%%%%%%%%%%%%%%%%%%%%%%%%%%%%%%%%%%%%%%%%%%%%%%%%%%%%%%%%%%%%%%%%%%%%%%%%%%%%
\begin{frame}[fragile]
   \frametitle{Ler uma matriz}
   Por padrão, o MATLAB lê os dados em um vetor. Se quisermos especificar o tamanho da saída dos dados, devemos acrescentar um argumento à função \acode{fscanf}:
   \begin{center}
      \code{A = \acode{fscanf}(arquivo,'\%f',[3 3])}
   \end{center}
   \begin{center}
      \begin{tikzpicture}
         \node[fill=pineglade] (caixa) {
           \begin{minipage}{8cm}
\begin{verbatim}
arquivo = fopen('matriz.txt');
A = fscanf(arquivo,'%f',[3 3]);
fclose(arquivo);
% A verdadeira matriz eh a transposta
% da matriz que foi lida:
A = A';
A
\end{verbatim}
           \end{minipage}
        };
      \end{tikzpicture}
   \end{center}
\end{frame}
%%%%%%%%%%%%%%%%%%%%%%%%%%%%%%%%%%%%%%%%%%%%%%%%%%%%%%%%%%%%%%%%%%%%%%%%%%%%%%%%%%%%%%%%%
\begin{frame}
   \frametitle{Ler uma matriz desconhecida}
   Se não sabemos o tamanho da matriz que está no arquivo, não podemos informar seu formato. Mas podemos contar quantos elementos foram lidos do arquivo:
   \begin{center}
      \code{[A,contador] = \acode{fscanf}(arquivo,'\%d')}
   \end{center}
   % Para ler o conteúdo do arquivo sem interpretar os valores, podemos usar o comando
   % \begin{center}
   %    \code{\acode{type}('nome.txt')}
   % \end{center}
\end{frame}
%%%%%%%%%%%%%%%%%%%%%%%%%%%%%%%%%%%%%%%%%%%%%%%%%%%%%%%%%%%%%%%%%%%%%%%%%%%%%%%%%%%%%%%%%
% \begin{frame}
%    \frametitle{Exercício}
%    Ler uma matriz de tamanho desconhecido (mas quadrada), e retransformá-la em sua forma original.
%    \vfill
%    \file{0.4\textwidth}{matrixquadrada.txt}{\code{1 23 3 4 5 6 7\\
% 4 5 6 68 4 6 3\\
% 7 10 9 5 345 3 2\\
% 2 4 4 7 5 3 2\\
% 67 434 2 4 6 7 66\\
% 3 65 8 43 11 4 0\\
% 1 67 3 788 3 5 8}}
% \end{frame}
%%%%%%%%%%%%%%%%%%%%%%%%%%%%%%%%%%%%%%%%%%%%%%%%%%%%%%%%%%%%%%%%%%%%%%%%%%%%%%%%%%%%%%%%%
% \begin{frame}[fragile]
%    \frametitle{Exercício: resposta}
%    \begin{center}
%       \begin{minipage}{5cm}
%          \begin{beamercolorbox}[sep=1em,wd=5cm]{block body}
%             \begin{center}
%                \code{lermatrizquadrada.m}
%             \end{center}
%          \end{beamercolorbox}
%       \end{minipage}
%    \end{center}
% \end{frame}
%%%%%%%%%%%%%%%%%%%%%%%%%%%%%%%%%%%%%%%%%%%%%%%%%%%%%%%%%%%%%%%%%%%%%%%%%%%%%%%%%%%%%%%%%
\begin{frame}
   \frametitle{Formatos avançados}
   Suponha que temos no nosso arquivo também o nome do campo de dados:
   \file{0.5\textwidth}{temperaturas.txt}{\vspace*{-0.3cm}\footnotesize{\code{
      \begin{tabular}{r r r r}
         Hora & 1, & Temperatura & 20.6\\
         Hora & 3, & Temperatura & 21.2\\
         Hora & 5, & Temperatura & 23.1\\
         Hora & 6, & Temperatura & 24.5\\
         Hora & 8, & Temperatura & 25.0\\
         Hora & 9, & Temperatura & 25.2\\
         Hora & 10, & Temperatura & 25.8
      \end{tabular}\vspace*{-0.5cm}
   }}}
   Para ler apenas os números desta tabela, usamos o comando
   \begin{center}
      \code{A = \acode{fscanf}(arquivo,'Hora \%d, Temperatura \%f\textbackslash n',[2 7])}
   \end{center}
\end{frame}
%%%%%%%%%%%%%%%%%%%%%%%%%%%%%%%%%%%%%%%%%%%%%%%%%%%%%%%%%%%%%%%%%%%%%%%%%%%%%%%%%%%%%%%%%
\begin{frame}
   \frametitle{Exemplo}
   Se não conhecemos quantos dados estão na lista (mas sabemos que são 2 por linha), podemos especificar 
   \begin{center}
      \small{\code{A = \acode{fscanf}(arquivo,'Hora \%d, Temperatura \%f\textbackslash n',[2 Inf])}}
   \end{center}
\end{frame}
%%%%%%%%%%%%%%%%%%%%%%%%%%%%%%%%%%%%%%%%%%%%%%%%%%%%%%%%%%%%%%%%%%%%%%%%%%%%%%%%%%%%%%%%%
\begin{frame}
   \frametitle{Exercício}
   Dado um arquivo com a tabela abaixo de nomes e idades, calcule a média de idades deste grupo.
   \begin{center}
      \begin{tabular}{l l}
         Antonio &12\\
         Bruno &20\\
         Caio &34\\
         Danilo &21\\
         Eder &45\\
         Fernando &78\\
         Gustavo &20
      \end{tabular}
   \end{center}
   Dica: para não ler o texto, e ler apenas as idades, podemos \emph{pular} o campo de texto com o comando
   \begin{center}
      \code{idades = \acode{fscanf}(arquivo,'\%\alert{*}s \%d\textbackslash n')}
   \end{center}
\end{frame}
%%%%%%%%%%%%%%%%%%%%%%%%%%%%%%%%%%%%%%%%%%%%%%%%%%%%%%%%%%%%%%%%%%%%%%%%%%%%%%%%%%%%%%%%%
\begin{frame}[fragile]
   \frametitle{Exercícios: resposta}
   \begin{center}
      \begin{tikzpicture}
         \node[fill=pineglade] (caixa) {
           \begin{minipage}{10cm}
\begin{verbatim}
arquivo = fopen('tabela.txt');
idades = fscanf(arquivo,'%*s %d\n');
fclose(arquivo);
% A funcao mean(vetor) calcula a media dos 
% valores do vetor.
media = mean(idades);
disp(['A media de idades do grupo eh ' ...
       num2str(media) ' anos.'])
\end{verbatim}
           \end{minipage}
        };
      \end{tikzpicture}
   \end{center}
\end{frame}
%%%%%%%%%%%%%%%%%%%%%%%%%%%%%%%%%%%%%%%%%%%%%%%%%%%%%%%%%%%%%%%%%%%%%%%%%%%%%%%%%%%%%%%%%
\begin{frame}
   \frametitle{\acode{xlsread}}

   Para ler arquivos de planilha gerados pelo Microsoft Excel, usamos
   \codigo{\ac & [dados,texto,resto] = \acode{xlsread}(arquivo)}
   % reads data from the first worksheet in
   %  the Microsoft Excel spreadsheet file named FILE and returns the numeric
   %  data in array NUM. Optionally, returns the text fields in cell array
   %  TXT, and the unprocessed data (numbers and text) in cell array RAW.
   \vfill
   Para ler os dados de uma planilha específica do arquivo, usamos
   \codigo{\ac & [dados,texto,resto] = \acode{xlsread}(arquivo,planilha)}
   \vfill
   Em sistemas com o Microsoft Excel instalado, pode-se usar
   \codigo{\ac & [dados,texto,resto] = \acode{xlsread}(arquivo,-1)}
   para abrir uma janela do Excel e selecionar os dados a serem importados interativamente.
 \end{frame}
%%%%%%%%%%%%%%%%%%%%%%%%%%%%%%%%%%%%%%%%%%%%%%%%%%%%%%%%%%%%%%%%%%%%%%%%%%%%%%%%%%%%%%%%%
% \begin{frame}
%    \frametitle{\code{tblread}}

%    O comando
%    \codigo{\ac & [dados,var,casos] = \acode{tblread}(nomedoarquivo,separador)}
%    lê os seguintes valores de um arquivo:
%    \begin{itemize}
%       \item \code{dados}: Matriz numérica com um valor para cada par variável-caso
%       \item \code{var}: Matriz de texto contendo os nomes das variáveis na primeira linha do arquivo
%       \item \code{casos}: Matriz de texto contendo os nomes dos casos na primeira coluna do arquivo
%    \end{itemize}
%    Os valores no arquivo devem estar separados pelo \code{separador} indicado.
% \end{frame}
%%%%%%%%%%%%%%%%%%%%%%%%%%%%%%%%%%%%%%%%%%%%%%%%%%%%%%%%%%%%%%%%%%%%%%%%%%%%%%%%%%%%%%%%%
\begin{frame}
  \frametitle{Escrita em arquivos: \code{save}}

  Para salvarmos alguma variável em um arquivo, podemos usar o comando

  \codigo{\ac & \acode{save}('arquivo.mat','variavel')}

  Porém, este comando salva o arquivo no formato MAT, que é um formato próprio do MATLAB, ilegível para humanos. Assim, para salvarmos em um arquivo texto simples, acrescentamos a opção \code{'-ascii'}.
  
  Exemplo:
    \codigo{\ac & dados = \acode{rand}(3,4);\\
      \ac & \acode{save}('dadosout.txt','dados','-ascii')}
\end{frame}
%%%%%%%%%%%%%%%%%%%%%%%%%%%%%%%%%%%%%%%%%%%%%%%%%%%%%%%%%%%%%%%%%%%%%%%%%%%%%%%%%%%%%%%%%
\begin{frame}
   \frametitle{Escrita}
   Para escrever em um arquivo existente, a sintaxe é similar, mas devemos avisar ao MATLAB que vamos escrever neste arquivo:
   \codigo{\ac & arquivo = \acode{fopen}('info.txt',\acode{'w'})\\
     \ac & \acode{fprintf}(arquivo,'\%d',1)\\
     \ac & \acode{fclose}(arquivo)}

   \alert{Obs.} As opções do comando \acode{fopen} são:
   \codigo{\ac & arquivo = \acode{fopen}('info.txt',\acode{'r'})\\
     \ac & arquivo = \acode{fopen}('info.txt',\acode{'w'})\\
     \ac & arquivo = \acode{fopen}('info.txt',\acode{'a'})}
\end{frame}
%%%%%%%%%%%%%%%%%%%%%%%%%%%%%%%%%%%%%%%%%%%%%%%%%%%%%%%%%%%%%%%%%%%%%%%%%%%%%%%%%%%%%%%%%
% \begin{frame}
%    \frametitle{Exemplo}
%    Escreva um vetor em um arquivo.
%    \vfill
%    \begin{center}
%       \begin{minipage}{5cm}
%          \begin{beamercolorbox}[sep=1em,wd=5cm]{block body}
%             \begin{center}
%                \code{escrevervetor.m}
%             \end{center}
%          \end{beamercolorbox}
%       \end{minipage}
%    \end{center}
% \end{frame}
%%%%%%%%%%%%%%%%%%%%%%%%%%%%%%%%%%%%%%%%%%%%%%%%%%%%%%%%%%%%%%%%%%%%%%%%%%%%%%%%%%%%%%%%%
% \begin{frame}
%    \frametitle{Exercício}
%    Escreva uma matriz em um arquivo.
%    \vfill
%    \begin{center}
%       \begin{minipage}{5cm}
%          \begin{beamercolorbox}[sep=1em,wd=5cm]{block body}
%             \begin{center}
%                \code{escrevermatriz.m}
%             \end{center}
%          \end{beamercolorbox}
%       \end{minipage}
%    \end{center}
% \end{frame}
%%%%%%%%%%%%%%%%%%%%%%%%%%%%%%%%%%%%%%%%%%%%%%%%%%%%%%%%%%%%%%%%%%%%%%%%%%%%%%%%%%%%%%%%%
% \begin{frame}
%    \frametitle{Exercício}
%    Escrever uma lista de textos armazenada em uma célula em um arquivo.
%    \vfill
%    \begin{center}
%       \begin{minipage}{5cm}
%          \begin{beamercolorbox}[sep=1em,wd=5cm]{block body}
%             \begin{center}
%                \code{escrevercelula.m}
%             \end{center}
%          \end{beamercolorbox}
%       \end{minipage}
%    \end{center}
% \end{frame}
%%%%%%%%%%%%%%%%%%%%%%%%%%%%%%%%%%%%%%%%%%%%%%%%%%%%%%%%%%%%%%%%%%%%%%%%%%%%%%%%%%%%%%%%%
% \begin{frame}
%    \frametitle{Extras}
%    \begin{itemize}
%       \item \acode{rewind}: posição do cursor
%       \item Reescrever arquivos
%    \end{itemize}
% \end{frame}
%%%%%%%%%%%%%%%%%%%%%%%%%%%%%%%%%%%%%%%%%%%%%%%%%%%%%%%%%%%%%%%%%%%%%%%%%%%%%%%%%%%%%%%%%
\begin{frame}
  \frametitle{}
  \begin{center}
    \alert{Métodos para Análise Estatística}
  \end{center}
\end{frame}
%%%%%%%%%%%%%%%%%%%%%%%%%%%%%%%%%%%%%%%%%%%%%%%%%%%%%%%%%%%%%%%%%%%%%%%%%%%%%%%%%%%%%%%%%
\begin{frame}
  \frametitle{Média aritmética simples}

  Podemos calcular o valor da média aritmética simples para um conjunto de números armazenados em um vetor \code{x} usando o comando 

  \codigo{\ac & \acode{mean}(x)}

  Para calcularmos a média aritmética simples \alert{de cada coluna} de uma matriz e armazenarmos essas médias em um vetor linha, podemos usar o comando

  \codigo{\ac & \acode{mean}(matriz)}

  Para calcularmos a média aritmética simples \alert{de cada linha} de uma matriz e armazenarmos o resultado em uma matriz coluna, usamos o comando

  \codigo{\ac & \acode{mean}(matriz,2)}
\end{frame}
%%%%%%%%%%%%%%%%%%%%%%%%%%%%%%%%%%%%%%%%%%%%%%%%%%%%%%%%%%%%%%%%%%%%%%%%%%%%%%%%%%%%%%%%%
\begin{frame}
   \frametitle{\code{trimmmean}}
 
   O comando 
   \codigo{\ac & m = \acode{trimmean}(X,pc)}

   calcula a média excluindo os $k$ maiores e menores valores de um vetor \code{X}, em que 
   \begin{equation*}
      k = \frac{n*\frac{pc}{100}}{2}
   \end{equation*} 
   e onde $n$ é o número de valores em \code{X}.
\end{frame}
%%%%%%%%%%%%%%%%%%%%%%%%%%%%%%%%%%%%%%%%%%%%%%%%%%%%%%%%%%%%%%%%%%%%%%%%%%%%%%%%%%%%%%%%%
\begin{frame}
   \frametitle{Mediana}

   Para calcularmos a mediana de um conjunto de dados armazenados em um vetor, usamos o comando

   \codigo{\ac & \acode{median}(x)}

   Para calcularmos a mediana das colunas de uma matriz, e retornar as medianas em um vetor linha, usamos o comando

   \codigo{\ac & \acode{median}(matriz)}
\end{frame}
%%%%%%%%%%%%%%%%%%%%%%%%%%%%%%%%%%%%%%%%%%%%%%%%%%%%%%%%%%%%%%%%%%%%%%%%%%%%%%%%%%%%%%%%%
\begin{frame}
  \frametitle{Desvio Padrão}

  Existem duas definições para o desvio padrão:
  \begin{center}
    \begin{tikzpicture}
      \node[fill=pineglade] (caixa) {
        \begin{minipage}{5.5cm}
          \vskip-.3cm
          \begin{equation}
            \label{eq:std1}
            s = \left( \frac{1}{n-1} \sum_{i=1}^n (x_i-\overline{x})^2\right)^{\frac{1}{2}}
          \end{equation}
        \end{minipage}
      };
    \end{tikzpicture}\\
    ou\\
    \vskip.2cm
    \begin{tikzpicture}
      \node[fill=pineglade] (caixa) {
        \begin{minipage}{5.5cm}
          \vskip-.3cm
          \begin{equation}
            \label{eq:std2}
            s = \left( \frac{1}{n} \sum_{i=1}^n (x_i-\overline{x})^2\right)^{\frac{1}{2}}
          \end{equation}
        \end{minipage}
      };
    \end{tikzpicture}
  \end{center}
onde
\begin{equation*}
  \overline{x} = \frac{1}{n} \sum_{i=1}^n x_i.
\end{equation*}
\end{frame}
%%%%%%%%%%%%%%%%%%%%%%%%%%%%%%%%%%%%%%%%%%%%%%%%%%%%%%%%%%%%%%%%%%%%%%%%%%%%%%%%%%%%%%%%%
\begin{frame}
  \frametitle{Desvio Padrão}

  Para calcularmos o desvio padrão usando a fórmula (1), usamos o comando

  \codigo{\ac & \acode{std}(x)}

  O resultado é a raiz quadrada da variância.

  Se quisermos calcular um vetor linha contendo o desvio padrão calculado para cada coluna de uma matriz, usamos

  \codigo{\ac & \acode{std}(matriz)}

  Se quisermos calcular o desvio padrão dos elementos de um vetor usando a fórmula (2), usamos

  \codigo{\ac & \acode{std}(x,1)}
\end{frame}
%%%%%%%%%%%%%%%%%%%%%%%%%%%%%%%%%%%%%%%%%%%%%%%%%%%%%%%%%%%%%%%%%%%%%%%%%%%%%%%%%%%%%%%%%
\begin{frame}
  \frametitle{Variância}

  Para calcularmos a variância dos elementos de um vetor, usamos o comando
  
  \codigo{\ac & \acode{var}(x)}

  Para calcularmos um vetor linha com as variâncias de cada coluna da matriz, usamos o comando

  \codigo{\ac & \acode{var}(matriz)}

  O comando \acode{var} normaliza os dados por $n-1$, se temos $n>1$ dados. Se desejamos normalizar por $n$, usamos o comando

  \codigo{\ac & \acode{var}(x,1)}
\end{frame}
%%%%%%%%%%%%%%%%%%%%%%%%%%%%%%%%%%%%%%%%%%%%%%%%%%%%%%%%%%%%%%%%%%%%%%%%%%%%%%%%%%%%%%%%%
\begin{frame}
   \frametitle{Covariância}

   Para calcularmos a matriz de covariância entre 2 variáveis de dados, usamos o comando

   \codigo{\ac & \acode{cov}(X)}

   Podemos ainda obter outras informações desta matriz:

   \codigo{\ac & \acode{diag}(\acode{cov}(X))}

   é o vetor de variâncias para cada coluna de dados (idem a \acode{var})

   \codigo{\ac & \acode{sqrt}(\acode{diag}(\acode{cov}(X)))}
   
   é desvio padrão (idem a \acode{std}).

   \code{X} pode ser um vetor ou uma matriz. Para uma matriz $m\times n$, a matriz de covariância é $n\times n$.
\end{frame}
%%%%%%%%%%%%%%%%%%%%%%%%%%%%%%%%%%%%%%%%%%%%%%%%%%%%%%%%%%%%%%%%%%%%%%%%%%%%%%%%%%%%%%%%%
\begin{frame}
   \frametitle{Coeficientes de Correlação}

   Se tivermos uma matriz em que cada coluna contém observações de uma variável, podemos calcular os coeficientes de correlação entre as variáveis desta matriz usando o comando

   \codigo{\ac & R = \acode{corrcoef}(X)}

   Os coeficientes vão de $-1$ (correlação negativa) até $1$ (correlação positiva). Valores próximos de $0$ indicam que não há correlação linear entre as variáveis.

   Se também quisermos saber o \emph{p-value} de cada correlação, usamos o comando

   \codigo{\ac & [R, P] = \acode{corrcoef}(X)}
\end{frame}
%%%%%%%%%%%%%%%%%%%%%%%%%%%%%%%%%%%%%%%%%%%%%%%%%%%%%%%%%%%%%%%%%%%%%%%%%%%%%%%%%%%%%%%%%
\begin{frame}
   \frametitle{Exemplo}

   Calcular a matriz de correlação e os \emph{p-values} entre as colunas da matriz \code{X}:
   \codigo{\ac & [R,P] = \acode{corrcoef}(X)}
   Encontrar todos os índices da matriz de correlação para os quais o \emph{p-value} é menor que $0.05$:
   \codigo{\ac & [i,j] = \acode{find}(p<0.05)}
\end{frame}
%%%%%%%%%%%%%%%%%%%%%%%%%%%%%%%%%%%%%%%%%%%%%%%%%%%%%%%%%%%%%%%%%%%%%%%%%%%%%%%%%%%%%%%%%
\begin{frame}
   \frametitle{Correlação (2)}

   Para encontrarmos a matriz de correlação entre variáveis e seus respectivos \emph{p-values}, também podemos usar a função \acode{corr}, com mais opções:

   \codigo{\ac & [RHO,PVAL] = \acode{corr}(X,Y,'nome',valor)}

   Exemplos:

   \begin{itemize}
      \item[\ac] \code{[RHO,PVAL] = \acode{corr}(X,Y,'type','Pearson')}
      \item[\ac] \code{[RHO,PVAL] = \acode{corr}(X,Y,'type','Kendall')}
      \item[\ac] \code{[RHO,PVAL] = \acode{corr}(X,Y,'rows','all')}
      \item[\ac] \code{[RHO,PVAL] = \acode{corr}(X,Y,'rows','complete')} : \alert{pula linhas com \code{NaN}!}
   \end{itemize}
\end{frame}
%%%%%%%%%%%%%%%%%%%%%%%%%%%%%%%%%%%%%%%%%%%%%%%%%%%%%%%%%%%%%%%%%%%%%%%%%%%%%%%%%%%%%%%%%
\begin{frame}
   \frametitle{Histograma}

   Um histograma pode ser criado com o comando
   \codigo{\ac & n = \acode{hist}(Y)}
   em que o vetor \code{Y} é distribuido em 10 caixas igualmente espaçadas, e \code{n} é o número de elementos em cada caixa.
   O comando
   \codigo{\ac & n = \acode{hist}(Y,nbins)}
   divide os dados em \code{nbins} caixas.
\end{frame}
%%%%%%%%%%%%%%%%%%%%%%%%%%%%%%%%%%%%%%%%%%%%%%%%%%%%%%%%%%%%%%%%%%%%%%%%%%%%%%%%%%%%%%%%%
\begin{frame}
  \frametitle{Box Plot}

  O comando
  \codigo{\ac & \acode{boxplot}(X)}
  cria um gráfico de caixas dos dados em \code{X}. Se \code{X} for uma matriz, existirá uma caixa por coluna; se \code{X} for um vetor, existirá apenas uma caixa.
  \vfill
  Em cada caixa:
  \begin{itemize}
     \item a marca central é a mediana;
     \item os limites da caixa representam o $25^{\circ}$ e o $75^{\circ}$ percentil;
     \item os marcadores externos sinalizam os pontos extremos dos dados (sem considerar \emph{outliers};
     \item os \emph{outliers}, se existirem, serão marcados individualmente no gráfico.
  \end{itemize}
\end{frame}
%%%%%%%%%%%%%%%%%%%%%%%%%%%%%%%%%%%%%%%%%%%%%%%%%%%%%%%%%%%%%%%%%%%%%%%%%%%%%%%%%%%%%%%%%
% \begin{frame}
%    \frametitle{Box Plot (2)}
%    \codigo{\ac & \acode{boxplot}(X,G)}
%    especifica uma ou mais variáveis de agrupamento \code{G}, produzindo uma caixa separada para cada conjunto de valores em \code{X} no mesmo grupo.
% \end{frame}
%%%%%%%%%%%%%%%%%%%%%%%%%%%%%%%%%%%%%%%%%%%%%%%%%%%%%%%%%%%%%%%%%%%%%%%%%%%%%%%%%%%%%%%%%
\end{document}